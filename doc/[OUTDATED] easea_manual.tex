\documentclass{book}
\usepackage{array}
\usepackage[francais]{babel}
\usepackage[utf8]{inputenc}
\begin{document}

\title{The EASEA manual}
\author{Equipe BFO \\
  \small Université de Strasbourg}
\maketitle


\setcounter{tocdepth}{1}
\tableofcontents
\section{This manual} % (fold)
\label{sec:introduction}
  \paragraph{} % (fold)
  \label{par:}

  This document is intended to programmer working on the EASEA platform and to
  everyone working with it. It contain the language and idioms description, the
  concept behind it and the documentation related to the compiler and its genetic
  algorithm library.
% paragraph  (end)
% section introduction (end)
\chapter{About EASEA}
\section{Introduction} % (fold)
\label{sec:Introduction}
\paragraph{} % (fold)
\label{par:}

EASEA and EASEA-CLOUD are Free Open Source Software (under GNU Affero v3 General Public License) developed by the SONIC (Stochastic Optimisation and Nature Inspired Computing) group of the BFO team at Université de Strasbourg. Through the Strasbourg Complex Systems Digital Campus, the platforms are shared with the UNESCO CS-DC UniTwin and E-laboratory on Complex Computational Ecosystems (ECCE).
% paragraph  (end)
\paragraph{} % (fold)
\label{par:}
  
EASEA (EAsy Specification of Evolutionary Algorithms) is an Artificial Evolution
platform that allows scientists with only basic skills in computer science to
implement evolutionary algorithms and to exploit the massive parallelism of
many-core architectures in order to optimize virtually any real-world problems
(continous, discrete, combinatorial, mixed and more (with Genetic Programming)),
typically allowing for speedups up to x500 on a \$3,000 machine, depending on the complexity of the evaluation function of the inverse problem to be solved.
% paragraph  (end)

\paragraph{} % (fold)
\label{par:}
  
Then, for very large problems, EASEA can also exploit computational ecosystems as it can parallelize (using an embedded island model) over loosely coupled heterogenous machines (Windows, Linux or Macintosh, with or without GPGPU cards, provided that they have internet access) a grid of computers or a Cloud.
% paragraph  (end)
% section Introduction (end)

\section{Capabilities and features} % (fold)
\label{sec:Capabilities}

  \subsection{The platform} % (fold)
  \label{sub:subsection name}
  \paragraph{} % (fold)
  \label{par:}
  
    Runs can be distributed over cluster of homogeneous AND heterogeneous machines.
    Distribution can be done locally on the same machine or over the internet (using a embedded island model).
    Parallelization over GPGPU cards leading to massive speedup (x100 to x1000).
    C++-like description language.
  % paragraph  (end)
  \paragraph{} % (fold)
  \label{par:}
  
  EASEA use CUDA to parallelize over GPGPU card. There is, as absurd as it sound, no
  parallelization over CPU at the time of redaction, but a working openMP prototype
  enters its final testing phase.
  % paragraph  (end)
  \paragraph{} % (fold)
  \label{par:}
    
  EASEA offer a high level of parametrization and the possibility to include large
  chunk of C/C++ code with little to no change at all.
  % paragraph  (end)
  % subsection subsection name (end)
    \subsection{Current genetic algorithm implementation} % (fold)
    \label{sub:current genetic algorithm implementation}
    \paragraph{} % (fold)
    \label{par:}
    \begin{itemize}
      \item Darwinian and baldwinian approach
      \item Genetic programming
      \item CMA-ES (Covariance Matrix Adaptation Evolution Strategy)
      \item Memetic approach 
    \end{itemize}
\\
    \paragraph{} % (fold)
    \label{par:}
    
    EASEA also offer out-of-the-box a wide range of selector: Max/Min deterministic,
    Max/Min Random, Max/Min tournament and MaxRoulette.
    % paragraph  (end)
    \paragraph{} % (fold)
    \label{par:}
    
    Three stopping criterion are possible : by generation, by time and by user
    defined control.
    % paragraph  (end)
\\
    The main advantage of EASEA 
    % paragraph  (end)
    % subsection current genetic algorithm implementation (end)
% section Capabilities (end)


\section{Support} % (fold)
\label{sec:Support}
In order to have a working grapher, java jre 1.6 is required. Without it, an error appears at the start of easea's compiled programs but can be safely ignored.
\\
EASEA had been compiled and tested with those following compilers:

    \begin{itemize}
      \item Gcc 4.4 to 4.8.2
      \item Clang 3.0 to 3.3
      \item Mingw-gcc 4.8.2
      \item CUDA SDK > 4.1
    \end{itemize}

% section Support (end)
\section{Workflow} % (fold)
\label{sec:Workflow}
\paragraph{} % (fold)
\label{par:}

The workflow of an easea program is the following:\\
easea code (.ez) \rightarrow easea compile \rightarrow C++ generated
source\rightarrow C++ compile\rightarrow executable 
% paragraph  (end)
\paragraph{} % (fold)
\label{par:}

The easea compiler generate high-performance C++ sources from the user's easea source
code and predefined templates.
% paragraph  (end)

% section Workflow (end)

\chapter{The language} % (fold)
\label{sec:the language}
 \section{EASEA defined sections} % (fold)
 \label{sub:EASEA defined sections}
 
\subsection{Genome specific fields}\label{genome-specific-fields}

\subsubsection{Genome Initialiser}\label{genome-initialiser}

\paragraph{Description}\label{description}
~\\
Uses the Genome EASEA variable
to access the individual's genome.

\paragraph{EASEA syntax}\label{easea-syntax}
~\\

\texttt{\textbackslash{}GenomeClass::initialiser~:}\\\texttt{~~~~~...}\\\texttt{\textbackslash{}end}

\paragraph{Example}\label{example}
~\\

\texttt{\textbackslash{}GenomeClass::initialiser~:}\\\texttt{~~for(int~i=0;~i}\texttt{(y)?(x):(y))~~~~~//Definition~of~the~MAX~global~function}\\\texttt{~~\#define~MIN(x,y)~((x)\textless{}(y)?(x):(y))~~~~~~//Definition~of~the~MIN~global~function}\\\texttt{~~\#define~PI~3.141592654~~~~~~~~~~~~~~~~~//Definition~of~the~PI~global~variable}\\\texttt{\textbackslash{}end}

\subsubsection{User functions}\label{user-functions}

\paragraph{Description}\label{description-1}
~\\

This is the section where the user can declare the different functions
he will need.

\paragraph{EASEA syntax}\label{easea-syntax-1}
~\\

\texttt{\textbackslash{}User~declarations~:}\\\texttt{~~~~~...}\\\texttt{\textbackslash{}end}

\paragraph{Example}\label{example-1}
~\\

\texttt{\textbackslash{}User~declarations~:}\\\texttt{~~float~gauss()}\\\texttt{~~/*~Generates~a~normally~distributed~random~value~with~variance~1~and~0~mean.}\\\texttt{~~~~~~Algorithm~based~on~}``\texttt{gasdev}''\texttt{~from~Numerical~recipes'~pg.~203.~*/}\\\texttt{~~\{}\\\texttt{~~~~int~iset~=~0;}\\\texttt{~~~~float~gset~=~0.0;}\\\texttt{~~~~float~v1~=~0.0,~v2~=~0.0,~r~=~0.0;}\\\texttt{~~~~float~factor~=~0.0;}\\\texttt{~~~~if~(iset)~\{}\\\texttt{~~~~~~iset~=~0;}\\\texttt{~~~~~~return~gset;}\\\texttt{~~~~\}}\\\texttt{~~~~else~\{~~~~}\\\texttt{~~~~~~do~\{}\\\texttt{~~~~~~~~~v1~=~(float)random(0.,1.)~*~2.0~-~1.0;}\\\texttt{~~~~~~~~~v2~=~(float)random(0.,1.)~*~2.0~-~1.0;}\\\texttt{~~~~~~~~~r~=~v1~*~v1~+~v2~*~v2;}\\\texttt{~~~~~~\}~}\\\texttt{~~~~~~while~(r~\textgreater{}~1.0);}\\\texttt{~~~~~~factor~=~sqrt~(-2.0~*~log~(r)~/~r);}\\\texttt{~~~~~~gset~=~v1~*~factor;}\\\texttt{~~~~~~iset~=~1;}\\\texttt{~~~~~~return~(v2~*~factor);}\\\texttt{~~~~\}}\\\texttt{~~\}}\\\texttt{\textbackslash{}end}

\subsubsection{User Classes}\label{user-classes}

\paragraph{Description}\label{description-2}
~\\

This is the section where the user will be able to declare:

\begin{enumerate}
\itemsep1pt\parskip0pt\parsep0pt
\item
  The genome of the individuals trough the GenomeClass class
\item
  Different classes that will be needed.
\end{enumerate}

The GenomeClass field has to be present and ideally not empty. All the
variables defined in this field (the genome) will be accessible in
several other fields using the 'Genome' variable. Other ``hidden''
variables can be accessed such as:

\begin{enumerate}
\item
  \begin{enumerate}
  \itemsep1pt\parskip0pt\parsep0pt
  \item
    The fitness (variable: fitness. Ex: Genome.fitness)
  \end{enumerate}
\end{enumerate}

\paragraph{EASEA syntax}\label{easea-syntax-2}
~\\

\texttt{\textbackslash{}User~Class~:}\\\texttt{~~~~~...}\\\texttt{~~GenomeClass\{}\\\texttt{~~~~~...}\\\texttt{~~\}}\\\texttt{\textbackslash{}end}

\paragraph{Example}\label{example-2}
~\\

\texttt{\textbackslash{}User~classes:}\\\texttt{~~Element~~~~~\{~}\\\texttt{~~~~int~~~~~Value;}\\\texttt{~~~~Element~*pNext;~\}}\\\texttt{~~GenomeClass~\{~}\\\texttt{~~~~Element~*pList;~}\\\texttt{~~~~int~~~~~Size;~~~\}}\\\texttt{\textbackslash{}end}

\subsubsection{User Makefile}\label{user-makefile}

\paragraph{Description}\label{description-3}
~\\

This section allows the user to add some flags to the compiler typically
to include some custom libraries.

Two flags of the Makefile are accessible to the user in this section:

\begin{enumerate}
\itemsep1pt\parskip0pt\parsep0pt
\item
  CPPFLAGS
\item
  LDFLAGS
\end{enumerate}

\paragraph{EASEA syntax}\label{easea-syntax-3}
~\\

\texttt{\textbackslash{}User~Makefile~options~:}\\\texttt{~~~~~...}\\\texttt{\textbackslash{}end}

\paragraph{Example}\label{example-3}
~\\

\texttt{\textbackslash{}User~Makefile~options~:}\\\texttt{~~CPPLAGS~+=~-llibrary}\\\texttt{~~LDFLAGS~+=~-I/path/to/include}\\\texttt{\textbackslash{}end}

\subsection{Miscellaneous fields}\label{miscellaneous-fields}

\subsubsection{Before everything else
function}\label{before-everything-else-function}

\paragraph{Description}\label{description-4}
~\\

This function will be called at the start of the algorithm before the
parent initialization.

\paragraph{EASEA syntax}\label{easea-syntax-4}
~\\

\texttt{\textbackslash{}Before~everything~else~function~:}\\\texttt{~~~~~...}\\\texttt{\textbackslash{}end}

\subsubsection{After everything else
function}\label{after-everything-else-function}

\paragraph{Description}\label{description-5}
~\\

This function will be called once the last generation has finished.

The population can be accessed.

\paragraph{EASEA syntax}\label{easea-syntax-5}
~\\

\texttt{\textbackslash{}After~everything~else~function~:}\\\texttt{~~~~~...}\\\texttt{\textbackslash{}end}

\subsubsection{At the beginning of each generation
function}\label{at-the-beginning-of-each-generation-function}

\paragraph{Description}\label{description-6}
~\\

This function will be called every generation before the reproduction
phase.

The population can be accessed.

\paragraph{EASEA syntax}\label{easea-syntax-6}
~\\

\texttt{\textbackslash{}At~the~~beginning~of~each~generation~function~:}\\\texttt{~~~~~...}\\\texttt{\textbackslash{}end}

\subsubsection{At the end of each generation
function}\label{at-the-end-of-each-generation-function}

\paragraph{Description}\label{description-7}
~\\

This function will be called at every generation after the printing of
generation statistics.

The population can be accessed.

\paragraph{EASEA syntax}\label{easea-syntax-7}
~\\

\texttt{\textbackslash{}At~the~end~of~each~generation~function~:}\\\texttt{~~~~~...}\\\texttt{\textbackslash{}end}

\subsubsection{At each generation before reduce
function}\label{at-each-generation-before-reduce-function}

\paragraph{Description}\label{description-8}
~\\

This function will be called at every generation before performing the
different population reductions.

The parent population AND the offspring population can be accessed
(merged into a single population).

\paragraph{EASEA syntax}\label{easea-syntax-8}
~\\

\texttt{\textbackslash{}At~every~generation~before~reduce~function~:}\\\texttt{~~~~~...}\\\texttt{\textbackslash{}end}

\subsection{Memetic specific field}\label{memetic-specific-field}

\subsubsection{Genome Optimiser}\label{genome-optimiser}

\paragraph{Description}\label{description-9}
~\\

The optimiser field is a Genome specific field. It is meant to contain
the function defining the way an individual will be locally optimized
\emph{n} times. The function will hence be called sequentially as many
times as the user desires for each individual.

EASEA gives the user two possibilities when designing their local
optimization function :

\begin{enumerate}
\itemsep1pt\parskip0pt\parsep0pt
\item
  The user can choose to design the function that will enhance the
  genome of their individuals only in which case the rest of the local
  optimizer (i.e. creating the local optimizing loop, checking if an
  individual has improved or not, storing temporary individuals, calling
  of the evaluation function, etc ...) will be taken care of by the
  EASEA memetic algorithm. The function will have to be called as many
  times as specified by the
  Number of optimisation iterations parameter.
\item
  The user can choose to write the complete local optimizer. This way,
  he will have the complete freedom to design a more complex and
  specific optimizer, but he will also have to deal with the creation of
  the local optimization loop, the management of temporary individuals,
  the calling of the evaluation function etc... The
  Number of optimisation iterations parameter will have to be set to 1 as the
  function desigend by the user will contain it's own optimization loop
  requiring it's own specific number of optimization iterations.
\end{enumerate}

\paragraph{EASEA syntax}\label{easea-syntax-9}
~\\

\texttt{\textbackslash{}GenomeClass::optimiser~:}\\\texttt{~~~~~...}\\\texttt{\textbackslash{}end}

\paragraph{Examples}\label{examples}
~\\

The two following examples will expose the two different ways the
optimizer field can be used. The first example will show a simple
\emph{mutation} function. The second example will show the design of a
complete local optimizer. Both examples are GPU compatible.

\begin{description}
\itemsep1pt\parskip0pt\parsep0pt
\item[Genome optimization only]
\end{description}

\texttt{\textbackslash{}GenomeClass::optimiser~:}\\\texttt{~float~pas~=~0.001;}\\\texttt{~Genome.x{[}currentIteration\%SIZE{]}+=pas;}\\\texttt{\textbackslash{}end}

This example shows a simple \emph{mutation} function that will add a
small variation to one of the genes of an individual. The call to this
function will be followed by a call to the evaluation function, and a
replacement process. If the modification of the genome has improved the
individual, it will replace the original one. This is being taken care
of by the EASEA memetic algorithm.

\begin{description}
\itemsep1pt\parskip0pt\parsep0pt
\item[Complete local optimizer]
\end{description}

\texttt{\textbackslash{}GenomeClass~:~:~optimiser~:~~~//~~Optimises~~~the~Genome~}\\\texttt{~float~pas=0.001;}\\\texttt{~float~fitnesstmp~=~Genome.fitness~;~}\\\texttt{~float~tmp{[}SIZE{]};~}\\\texttt{~int~index~=~0;}\\\texttt{~for(int~i=0;~i}\texttt{fitness}

\subsubsection{Best individual}\label{best-individual}

Returns a pointer to the best individual found to this point. All the
genome field can be accessed as well as the fitness field. All the
genome fields can be modified.

\paragraph{EASEA syntax}\label{easea-syntax-10}
~\\

\texttt{bBest~("-\textgreater{}"~operator~to~access~variables)}

\subsubsection{Genome}\label{genome}

Returns the genome of an individual (can only be used in genome specific
EASEA subsections such as
mutation,
initialisation,
evaluation and
. All the fields
can be modified.

\paragraph{EASEA syntax}\label{easea-syntax-11}
~\\

\texttt{Genome~("."~operator~to~access~variables)}

\subsubsection{Parent 1}\label{parent-1}

Returns the genome of the first selected parent. Can only be used in the
crossover genome
subsection.

\paragraph{EASEA syntax}\label{easea-syntax-12}
~\\

\texttt{parent1}

\subsubsection{Parent 2}\label{parent-2}

Returns the genome of the second selected parent. Can only be used in
the crossover genome
subsection.

\paragraph{EASEA syntax}\label{easea-syntax-13}
~\\

\texttt{parent2}

\subsubsection{Child}\label{child}

Returns the genome of the newly created individual. Can only be used in
the crossover genome
subsection.

\paragraph{EASEA syntax}\label{easea-syntax-14}
~\\

\texttt{child}

Here is a quick presentation of the various parameters that can be found
and used in a .ez file



\section{EASEA defined functions}
\subsection{Random Number Generators}\label{random-number-generators}

\subsubsection{TossCoin}\label{tosscoin}

\paragraph{Description}\label{description}

Simulates the toss of a coin. There are two different definitions of the
tossCoin function:

\begin{enumerate}
\itemsep1pt\parskip0pt\parsep0pt
\item
  A simple toss of a coin.
\item
  A biased toss of a coin.
\end{enumerate}

\paragraph{EASEA syntax}\label{easea-syntax}

\texttt{bool~tossCoin()~~~~~~~~~~~~~~~~//~SImple~tosscoin}

\texttt{bool~tossCoin(float~bias)~~//~Biased~tossCoin}

\paragraph{Example}\label{example}

\texttt{if(~tossCoin(0.1))\{}\\\texttt{~~~~...}\\\texttt{\}}

\subsubsection{Random}\label{random}

\paragraph{Description}\label{description-1}

Generates a random number. There are several definitions to the random
function:

\begin{enumerate}
\itemsep1pt\parskip0pt\parsep0pt
\item
  Random function with Min Max Boundary
\item
  Random function with Max Boundary only
\end{enumerate}

In the case of a Max boundary only, the Min boundary will be 0.

\paragraph{EASEA syntax}\label{easea-syntax-1}

\texttt{int~random(~int~Min,~int~Max)}

\texttt{int~random(~int~Max)}

\texttt{float~random(~float~Min,~float~Max)}

\texttt{double~random(~double~Min,~double~Max)}


\section{EASEA defined variables}
\subsection{Current Generation}\label{current-generation}

Return the current generation number. This variable can be modified.

\subsubsection{EASEA syntax}\label{easea-syntax-2}

\texttt{currentGeneration}

\subsection{Number of Generation}\label{number-of-generation}

Returns the generation limit. This variable can be modified.

\subsubsection{EASEA syntax}\label{easea-syntax-3}

\texttt{NB\_GEN}

\subsection{Population Size}\label{population-size}

Returns the size of the population. This variable cannot be modified.

\subsubsection{EASEA syntax}\label{easea-syntax-4}

\texttt{POP\_SIZE}

\subsection{Mutation Probability}\label{mutation-probability}

Returns the mutation probability. This variable can be modified.

\subsubsection{EASEA syntax}\label{easea-syntax-5}

\texttt{MUT\_PROB}

\subsection{Crossover Probability}\label{crossover-probability}

Returns the crossover probability. This variable can be modified.

\subsubsection{EASEA syntax}\label{easea-syntax-6}

\texttt{XOVER\_PROB}

\subsection{Minimise}\label{minimise}

Returns a boolean indicating of the algorithm performs a minimization of
not. Returns true if minimizing.

\subsubsection{EASEA syntax}\label{easea-syntax-7}

\texttt{MINIMISE}

\subsection{Population}\label{population}

Returns a pointer to the main population. This variable cannot be used
everywhere. If misused, it can provoque unexpected behaviours or compile
errors.

\subsubsection{EASEA syntax}\label{easea-syntax-8}

\texttt{pPopulation~({[}i{]}~to~access~individuals)}

\subsection{fitness}\label{fitness}

\texttt{pPopulation{[}i{]}-\textgreater{}fitness}

\subsection{Best individual}\label{best-individual}

Returns a pointer to the best individual found to this point. All the
genome field can be accessed as well as the fitness field. All the
genome fields can be modified.

\subsubsection{EASEA syntax}\label{easea-syntax-9}

\texttt{bBest~("-\textgreater{}"~operator~to~access~variables)}

\subsection{Genome}\label{genome}

Returns the genome of an individual (can only be used in genome specific
EASEA sections such as
\href{EASEA defined sections\#Genome_Mutation}{mutation},
\href{EASEA defined sections\#Genome_Initialiser}{initialisation},
{[}{[}EASEA defined sections\#Genome Evaluation\textbar{}evaluation and
\href{EASEA defined sections\#Genome_Display}{display}). All the fields
can be modified.

\subsubsection{EASEA syntax}\label{easea-syntax-10}

\texttt{Genome~("."~operator~to~access~variables)}

\subsection{Parent 1}\label{parent-1}

Returns the genome of the first selected parent. Can only be used in the
\href{EASEA defined sections\#Genome_Crossover}{crossover} genome
section.

\subsubsection{EASEA syntax}\label{easea-syntax-11}

\texttt{parent1}

\subsection{Parent 2}\label{parent-2}

Returns the genome of the second selected parent. Can only be used in
the \href{EASEA defined sections\#Genome_Crossover}{crossover} genome
section.

\subsubsection{EASEA syntax}\label{easea-syntax-12}

\texttt{parent2}

\subsection{Child}\label{child}

Returns the genome of the newly created individual. Can only be used in
the \href{EASEA defined sections\#Genome_Crossover}{crossover} genome
section.

\subsubsection{EASEA syntax}\label{easea-syntax-13}

\texttt{child}


\section{EASEA defined parameters}
\subsection{Basic Parameters}\label{basic-parameters}

\paragraph{Number of generations}\label{number-of-generations}
~\\

\textbf{Description :}\\Gives the maximum number of generations during
wich the evolutionary algorithm will run.

\textbf{EASEA syntax :}

\texttt{Number~of~generations:}

\textbf{Values :} Integer strictly over 0.

\textbf{Command line syntax :}

\texttt{-{}-nbGen=}

\textbf{Values :} Integer strictly over 0.

\paragraph{Time limit}\label{time-limit}
~\\

\textbf{Description :}\\Sets the maximum amount of time in seconds
during wich the evolutionary algorithm will be allowed to run. Setting
this parameter to 0 deactivates the time limit.

\textbf{EASEA syntax :}

\texttt{Time~limit~:}

\textbf{Values :} Positive Integer.

\textbf{Command line syntax :}

\texttt{-{}-timeLimit=}

\textbf{Values :} Positive Integer.

\paragraph{Population size}\label{population-size}
~\\

\textbf{Description :}\\Sets the size of the population that will be
evolved.

\textbf{EASEA syntax :}

\texttt{Population~size~:}

\textbf{Values :} Integer strictly over 0.\\ \textbf{Command line syntax
:}

\texttt{-{}-popSize=}

\textbf{Values :} Integer strictly over 0.\\

\paragraph{Offspring size}\label{offspring-size}
~\\

\textbf{Description :}\\Sets the number of individual that will be
produced trough evolutionary crossover and/or mutation.

\textbf{EASEA syntax :}

\texttt{Offspring~size~:}

\textbf{Values :} Integer strictly over 0.\\ \textbf{Command line syntax
:}

\texttt{-{}-nbOffspring=}

\textbf{Values :} Integer strictly over 0.\\

\paragraph{Mutation probability}\label{mutation-probability}
~\\

\textbf{Desciption :}\\This parameter determines the probability that an
individual will be mutated during it's creation process.

\textbf{EASEA syntax :}

\texttt{Mutation~probability~:}

\textbf{Values :} Real number between 0.0 and 1.0 .

\textbf{Command line syntax :}\\This parameter is not avaible in command
line yet.

\paragraph{Crossover probability}\label{crossover-probability}
~\\

\textbf{Description :}\\This parameter determines the probability that
an individual will be the result of a crossover during it's creation
process.

\textbf{EASEA syntax :}

\texttt{Crossover~probability~:}

\textbf{Values :} Real number between 0.0 and 1.0 .

\textbf{Command line syntax :}\\This parameter is not avaible in command
line yet.

\paragraph{Evaluator goal}\label{evaluator-goal}
~\\

\textbf{Description :}\\This parameter will set the goal of the
evolutionary algorithm.

\textbf{EASEA syntax :}

\texttt{Evaluator~goal~:}

\textbf{Values :} \emph{minimis/ze} or \emph{maximis/ze}

\textbf{Command line syntax :}\\This parameter is too fundamental to the
behaviour of the algorithm to be changed in command line.

\paragraph{Selection operator}\label{selection-operator}
~\\

\textbf{Description :}\\This parameter decides of the way individuals of
the parent population will be selected to create the new individuals.
Several operator are avaiable in EASEA:

Tournament (need a selection pressure)

Deterministic

Roulette (only when Evaluator goal = maximise)

Random

When the selection pressure is an integer over 0, the best individual
from the \emph{n} will be selected.\\When the selection pressure is a
real number between 0.0 and 1.0, the best individual of 2 will be
selection with the probability \emph{p} given by the selection pressure.

\textbf{EASEA syntax :}

\texttt{Selection~operator:~OPERATOR~SELECTION\_PRESSURE}

\textbf{Values for operators :} Tournament Deterministic Roulette
Random\\\textbf{Values for selection pressure :} Integer strictly over 0
\textbf{or} Real number between 0.0 and 1.0 .

\textbf{Compile line syntax :}

\texttt{-{}-selectionOperator=}\\\texttt{-{}-selectionPressure=}

\textbf{Values for operators :} Tournament Deterministic Roulette
Random\\\textbf{Values for selection pressure :} Integer strictly over 0
\textbf{or} Real number between 0.0 and 1.0 .

\paragraph{Surviving parents}\label{surviving-parents}
~\\

\textbf{Description :}\\This parameter will determine the number of
children that will be participating in the run for the next generation
(IL VA FALLOIR MODIFIER CETTE DESCRITPION)

\textbf{EASEA syntax :}

\texttt{Surviving~offspring:}

\textbf{Values :} Integer strictly positive \textbf{or} Percentage
(example: 100\%)

\textbf{Command line syntax :}

\texttt{-{}-survivingOffspring=}

\textbf{Values :} Integer strictly positive \textbf{or} Real number
value between 0.0 and 1.0

\paragraph{Reduce parents operator}\label{reduce-parents-operator}
~\\

\textbf{Description :}\\This parameter decides of the way individuals of
the parent population will be selected to participate in the next
selection process. Several operator are avaiable in EASEA:

Tournament (need a selection pressure)

Deterministic

Roulette (only when Evaluator goal = maximise)

Random

When the selection pressure is an integer over 0, the best individual
from the \emph{n} will be selected.\\When the selection pressure is a
real number between 0.0 and 1.0, the best individual of 2 will be
selection with the probability \emph{p} given by the selection pressure.

\textbf{EASEA syntax :}

\texttt{Reduce~parents~operator:~OPERATOR~SELECTION\_PRESSURE}

\textbf{Values for operators :} Tournament Deterministic Roulette
Random\\\textbf{Values for selection pressure :} Integer strictly over 0
\textbf{or} Real number between 0.0 and 1.0 .

\textbf{Compile line syntax :}

\texttt{-{}-reduceParentsOperator=}\\\texttt{-{}-reduceParentsPressure=}

\textbf{Values for operators :} Tournament Deterministic Roulette
Random\\\textbf{Values for selection pressure :} Integer strictly over 0
\textbf{or} Real number between 0.0 and 1.0 .

\paragraph{Reduce offspring
operator}\label{reduce-offspring-operator}
~\\

\textbf{Description :}\\This parameter decides of the way individuals of
the offspring population will be selected to participate in the next
selection process. Several operator are avaiable in EASEA:

Tournament (need a selection pressure)

Deterministic

Roulette (only when Evaluator goal = maximise)

Random

When the selection pressure is an integer over 0, the best individual
from the \emph{n} will be selected.\\When the selection pressure is a
real number between 0.0 and 1.0, the best individual of 2 will be
selection with the probability \emph{p} given by the selection pressure.

\textbf{EASEA syntax :}

\texttt{Reduce~offspring~operator:~OPERATOR~SELECTION\_PRESSURE}

\textbf{Values for operators :} Tournament Deterministic Roulette
Random\\\textbf{Values for selection pressure :} Integer strictly over 0
\textbf{or} Real number between 0.0 and 1.0 .

\textbf{Compile line syntax :}

\texttt{-{}-reduceOffspringOperator=}\\\texttt{-{}-reduceOffspringPressure=}

\textbf{Values for operators :} Tournament Deterministic Roulette
Random\\\textbf{Values for selection pressure :} Integer strictly over 0
\textbf{or} Real number between 0.0 and 1.0 .

\paragraph{Final reduce operator}\label{final-reduce-operator}
~\\

\textbf{Description :}\\This parameter decides of the way individuals
will be selected to participate in the next generation. Several operator
are avaiable in EASEA:

Tournament (need a selection pressure)

Deterministic

Roulette (only when Evaluator goal = maximise)

Random

When the selection pressure is an integer over 0, the best individual
from the \emph{n} will be selected.\\When the selection pressure is a
real number between 0.0 and 1.0, the best individual of 2 will be
selection with the probability \emph{p} given by the selection pressure.

\textbf{EASEA syntax :}

\texttt{Final~reduce~operator:~OPERATOR~SELECTION\_PRESSURE}

\textbf{Values for operators :} Tournament Deterministic Roulette
Random\\\textbf{Values for selection pressure :} Integer strictly over 0
\textbf{or} Real number between 0.0 and 1.0 .

\textbf{Compile line syntax :}

\texttt{-{}-reduceFinalOperator=}\\\texttt{-{}-reduceFinalPressure=}

\textbf{Values for operators :} Tournament Deterministic Roulette
Random\\\textbf{Values for selection pressure :} Integer strictly over 0
\textbf{or} Real number between 0.0 and 1.0 .

\paragraph{Elitism}\label{elitism}
~\\

\textbf{Description :}\\This parameter determines the type of elitism of
the evolutionary algorithm.

\textbf{EASEA syntax :}

\texttt{Elitism~:}

\textbf{Values :} Strong \textbf{or} Weak

\textbf{Compile line syntax :}

\texttt{-{}-{}-eliteType=}

\textbf{Values :} 1 (for Strong) \textbf{or} 0 (for Weak)

\paragraph{Elite}\label{elite}
~\\

\textbf{Description :}\\This parameter sets the number of individuals
that are going to be ``elites''. Set this parameter to 0 to deactivate
the elitism

\textbf{EASEA syntax :}

\texttt{Elite~:}

\textbf{Values :} Positive Integer. 0 to deactive elitism.

\textbf{Command line syntax :}

\texttt{-{}-elite=}

\textbf{Values :} Positive Integer. 0 to deactive elitism.

\subsection{Remote Island Model
Parameters}\label{remote-island-model-parameters}

\paragraph{Remote island model}\label{remote-island-model}
~\\

\textbf{Description :}\\Boolean that will activate the migration of
individuals towards other EASEA instances of the same problem that are
located on remote machines. The address of the machines has to be
specified in the \emph{IP file} parameter.

\textbf{EASEA syntax :}

\texttt{Remote~island~model~:}

\textbf{Values :} true \textbf{or} false.

\textbf{Command line syntax :}

\texttt{-{}-{}-remoteIslandModel=}

\textbf{Values :} 1 (for true) \textbf{or} 0 (for false).

\paragraph{IP file}\label{ip-file}
~\\

\textbf{Description :}\\This parameter contains the path to the File
containing the IP addresses and port of all the remote machines that can
receive individuals. This file needs to be structure as
such:\\a1.b1.c1.d1:p1\\a2.b2.c2.d2:p2\\...\\an.bn.cn.dn:pn\\No blank
line is allowed at the end of the file.\\An individual can be send to a
machine that isn't active.\\The IP address and port of the local machine
can be present in the file. EASEA will recognize and ignore it.\\If
there are no IP addresses in the file, or if only the local machine's
address is present in that file, the \emph{Remote island model}
Parameter will be set to \emph{false}.

\textbf{EASEA syntax :}

\texttt{IP~file~:}

\textbf{Values :} The path to the file containing the IP addresses
(example: ./ip.txt)

\textbf{Command line syntax :}

\texttt{-{}-ipFile=}

\textbf{Values :} The path to the file containing the IP addresses
(example: ./ip.txt)

\paragraph{Server Port}\label{server-port}
~\\

\textbf{'Description :}\\This parameter sets the port that will be used
by the island server to listen for the arrival of newcomers.

\textbf{EASEA syntax :}

\texttt{Server~port~:}

\textbf{Values :} A valid port number.

\textbf{Command line syntax :}

\texttt{-{}-{}-serverPort=}

\textbf{Values :} A valid port number.

\paragraph{Migration probability}\label{migration-probability}
~\\

\textbf{Description :}\\This parameter sets the probability for and
individual to migrate to a remote machine every generation.

\textbf{EASEA syntax :}

\texttt{Migration~probability~:}

\textbf{Values :} A real number between 0.0 and 1.0 .

\textbf{Command line syntax :}

\texttt{-{}-migrationProbability=}

\textbf{Values :} A real number between 0.0 and 1.0 .

\subsection{Memetic Parameters}\label{memetic-parameters}

\paragraph{Number of optimisation
iterations}\label{number-of-optimisation-iterations}
~\\

\textbf{Description :}\\This parameters gives the number of iterations
the local search algorithm will perform.

\textbf{EASEA syntax :}

\texttt{Number~of~optimisation~iterations~:}

\textbf{Values :} A positive integer.

\textbf{Command line syntax :}

\texttt{-{}-optimiseIterations=}

\textbf{Values :} A positive integer.

\paragraph{Baldwinism}\label{baldwinism}
~\\

\textbf{Description :}\\This parameter will set the behaviour of the
memetic algorithm. Setting it to \emph{true} will turn the algorithm
into a \textbf{Baldwinian} memetic algorithm. Setting it to \emph{false}
will turn the algorithm into a \textbf{Lamarckian} memetic algorithm.
This parameter's default value is \emph{true}.

\textbf{EASEA syntax :}

\texttt{Baldwinism~:}

\textbf{Values :} true \textbf{or} false.

\textbf{Command line syntax :}

\texttt{-{}-baldwinism=}

\textbf{Values :} 1 (for true) \textbf{or} 0 (for false).

\subsection{Miscalleous Parameters}\label{miscalleous-parameters}

\paragraph{Print stats}\label{print-stats}
~\\

\textbf{Description :}\\This parameter set to \emph{true} will show
different stats every generations:

Generation number

Time passed since the start of the algorithm

Number of evaluations completed

Fitness of the best individual found

Avegrage fitness

Standard deviation of the fitness

This parameter's default value is \emph{true}.

\textbf{EASEA syntax :}

\texttt{Print~stats~:}

\textbf{Value :} true \textbf{or} false.

\textbf{Command line syntax :}

\texttt{printStats=}

\textbf{Values :} 1 (for true) \textbf{or} 0 (for false).

\paragraph{Plot stats}\label{plot-stats}
~\\

\textbf{Description}\\This parameter set to \emph{true} will open a
\textbf{gnuplot} process, that will plot the curves representing the
evolution of:

The best fitness

The avegage fitness

The standard deviation of the fitness

To be able to use this parameter, gnuplot has to be installed on the
machine. This parameter is only avaiable for Linux users.\\The ploting
of the curves will happen simultaniously to the evolution. At the end of
the evolution, the resulting curves will be saved in a \textbf{.png}
file. It's name will be the same as the project name. This parameter's
default value is \emph{false}.

\textbf{EASEA syntax :}

\texttt{Plot~stats~:}

\textbf{Value :} true \textbf{or} false.

\textbf{Command line syntax :}

\texttt{plotStats=}

\textbf{Values :} 1 (for true) \textbf{or} 0 (for false).

\paragraph{Generate csv stats file}\label{generate-csv-stats-file}
~\\

\textbf{Description :}\\This parameter set to \emph{true} will save the
different stats shown with the
Print stats parameter to a
\textbf{.csv} file at the end of the evolution. The file name will be
the same as the project name. This parameter's default value is
\emph{false}.

\textbf{EASEA syntax :}

\texttt{Generate~csv~stats~file~:}

\textbf{Values :} true \textbf{or} false.

\textbf{Command line syntax :}

\texttt{-{}-generateCSV=}

\textbf{Values :} 1 (for true) \textbf{or} 0 (for false).

\paragraph{Generate gnuplot script}\label{generate-gnuplot-script}
~\\

\textbf{Description :}\\This parameter set to \emph{true} will create a
gnuplot script that will allow to plot the evolution in a \textbf{.png}
file which, name will be the same as the project name. To plot the
evolution, the script will save the stats in a \textbf{.dat} file. This
parameter's default value is \emph{false}.

\textbf{EASEA syntax :}

\texttt{Generate~gnuplot~script~:}

\textbf{Values :} true \textbf{or} false.

\textbf{Command line syntax :}

\texttt{-{}-generateGnuplotScript=}

\textbf{Values :} 1 (for true) \textbf{or} 0 (for false).

\subsubsection{Generate R script}\label{generate-r-script}

\textbf{Description :}\\This parameter set to \emph{true} will create a
R script that will allow to plot the evolution in a \textbf{.png} file,
which name will be the same as the project name. To plot the evolution,
the script will save the stats in a \textbf{.csv} file. This parameter's
default value is \emph{false}.

\textbf{EASEA syntax :}

\texttt{Generate~R~script~:}

\textbf{Values :} true \textbf{or} false.

\textbf{Command line syntax :}

\texttt{-{}-generateRScript=}

\textbf{Values :} 1 (for true) \textbf{or} 0 (for false).

\subsubsection{Save population}\label{save-population}

\textbf{Description :}\\This parameter set to \emph{true} will save the
final population in a \textbf{.pop} file, which name will be the same as
the project name. The population will be in a EASEA serialized form.
This parameter's default value is \emph{false}.

\textbf{EASEA syntax :}

\texttt{Save~population~:}

\textbf{Values :} true \textbf{or} false.

\textbf{Command line syntax :}

\texttt{-{}-savePopulation=}

\textbf{Values :} 1 (for true) \textbf{or} 0 (for false).

\subsubsection{Start from file}\label{start-from-file}

\textbf{Description :}\\This parameter set to \emph{true} will
initialize the population using the population saved the \textbf{.pop}
file that is in the current folder. The file has to have the save name
as the project. For this initialization to work, the individuals
contained in the file has to have the exact same Genome, and the number
of individuals has to be equal or greater to the population size.\\The
individuals have to be in the EASEA serialization form.\\This
parameter's default value is \emph{false}.

\textbf{EASEA syntax :}

\texttt{Start~from~file~:}

\textbf{Values :} true \textbf{or} false.

\textbf{Command line syntax :}

\texttt{-{}-saveFromFile=}

\textbf{Values :} 1 (for true) \textbf{or} 0 (for false).

\subsubsection{Print initial population}\label{print-initial-population}

\textbf{Description :}\\This parameter allow to print the initial
population. This parameter's default value is 0.

\textbf{EASEA syntax :}\\This is a command line exclusive parameter.

\textbf{Command line syntax :}

\texttt{-{}-printInitialPopulation=}

\textbf{Values :} 1 (for true) \textbf{or} 0 (for false).

\subsubsection{Print final population}\label{print-final-population}

\textbf{Description :}\\This parameter allow to print the final
population. This parameter's default value is 0.

\textbf{EASEA syntax :}\\This is a command line exclusive parameter.

\textbf{Command line syntax :}

\texttt{-{}-printFinalPopulation=}

\textbf{Values :} 1 (for true) \textbf{or} 0 (for false).

\subsection{User defined parameters}\label{user-defined-parameters}

EASEA defines 5 parameters that are left free to be used by the ".ez"
programmer.

\textbf{User parameters :}

\texttt{~-{}-u1~arg~~~~~~~~~~~~~~~~~~~~~~User~defined~parameter~1}\\\texttt{~-{}-u2~arg~~~~~~~~~~~~~~~~~~~~~~User~defined~parameter~2}\\\texttt{~-{}-u3~arg~~~~~~~~~~~~~~~~~~~~~~User~defined~parameter~3}\\\texttt{~-{}-u4~arg~~~~~~~~~~~~~~~~~~~~~~User~defined~parameter~4}\\\texttt{~-{}-u5~arg~~~~~~~~~~~~~~~~~~~~~~User~defined~parameter~5}

\textbf{Types :} u1 and u2 are string variables, u3 to u5 are integers.

'''Usage Exemple : ''' In the .ez source code

\texttt{~~int~a~=~setVariable(}``\texttt{u3}''\texttt{,b);}

\emph{'where : \textbf{\\}``a''}' is the variable set to the argument or
default value.\\\textbf{``u3''} is the argument name.\\\textbf{``b''} is
the default integer value, if the u3 argument has not been set.

\chapter{Developer documentation} % (fold)
\label{sec:Developer documentation}
\section{Compiler} % (fold)
\label{sub:Compiler}
   \subsection{Parsing the language} % (fold)
   \label{sub:subsection name}
   \paragraph{} % (fold)
   \label{par:}
   The compile does its parsing using lex and YACC, and more specifically the alex
   and AYACC implementation from bumble-bee software.\\
   A good enhancement is to move from alex and AYACC to flex and Bison. The main
   reason is that the ayacc/alex compilers only work on windows and there is some licenses
   fuzziness.

   The AYACC "compiler" generated the "EaseaParse.cpp" file from "EaseaParse.y" .
   The alex "compiler" generated the "EaseaLex.cpp" file from "Easealex.l" .

   Those file don't need to be re-generated at each compilation, only when changed.
   The correct command through wine are:
   
   wine ~/.WINE/drive\_c/Program\\ Files/Parser\\ Generator/BIN/ayacc.exe \$< -Tcpp -d
   wine ~/.wine/drive\_c/Program\\ Files/Parser\\ Generator/BIN/ALex.exe \$< -Tcpp -i
   
   \paragraph{Note} % (fold)
   \label{par:Note}
   When using the urrent version of ayacc or alex some problems arise. The generated
   code classes names changed leading to multiple error such as no "class
   YYPARSERNAME found" because the rest of the easea compiler defined this class
   name as YYPARSENAME.\\
   The only workaround it to vimdiff the files and relie on the compiler error to
   manually fix the definitions problems.
   % paragraph Note (end)
   % paragraph  (end)
   % subsection subsection name (end)
% subsection Compiler (end)
  \section{EASEA library} % (fold)
  \paragraph{} % (fold)
  \label{par:}
  The idea behind easea is the following: the EASEA compiler wrap predefined
  sections in class and function, and by polyphormism and pointer manipulation, are
  exploitable by evolutionary engine.

  The evolutionary engine is implemented as the evolutianary library, the libeasea.

  % paragraph  (end)
  \label{sub:EASEA library}
   \subsection{Main evolutionary loop} % (fold)
   \label{sub:subsection name}
   
   % subsection subsection name (end)
  % subsection EASEA library (end)
% section Developer documentation (end)
% subsection EASEA defined variable (end)

\section{Example} % (fold)
\label{sec:example}

% section example (end)

\section{Additional ressources} % (fold)
\label{sec:Additional}
Additional ressource can be found on the project wiki : and repository 
% section Additional (end)

\end{document}
\end{book}
