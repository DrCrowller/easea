\documentclass{article}
\usepackage{array}
\usepackage[francais]{babel}
\usepackage[utf8]{inputenc}
\begin{document}

\title{The EASEA manual}
\author{Equipe BFO \\
  \small Université de Strasbourg}
\maketitle


\section{This manual} % (fold)
\label{sec:introduction}
  \paragraph{} % (fold)
  \label{par:}

  This document is intended to programimer working on the EASEA platform and to
  everyone working with it. It contain the language and idioms description, the
  concept behind it and the documentation related to the compiler and its genetic
  algorithm library.
% paragraph  (end)
% section introduction (end)

\section{Introduction} % (fold)
\label{sec:Introduction}
\paragraph{} % (fold)
\label{par:}

EASEA and EASEA-CLOUD are Free Open Source Software (under GNU Affero v3 General Public License) developed by the SONIC (Stochastic Optimisation and Nature Inspired Computing) group of the BFO team at Université de Strasbourg. Through the Strasbourg Complex Systems Digital Campus, the platforms are shared with the UNESCO CS-DC UniTwin and E-laboratory on Complex Computational Ecosystems (ECCE).

EASEA (EAsy Specification of Evolutionary Algorithms) is an Artificial Evolution
platform that allows scientists with only basic skills in computer science to
implement evolutionary algorithms and to exploit the massive parallelism of
many-core architectures in order to optimize virtually any real-world problems
(continous, discrete, combinatorial, mixed and more (with Genetic Programming)),
typically allowing for speedups up to x500 on a \$3,000 machine, depending on the complexity of the evaluation function of the inverse problem to be solved.

Then, for very large problems, EASEA can also exploit computational ecosystems as it can parallelize (using an embedded island model) over loosely coupled heterogenous machines (Windows, Linux or Macintosh, with or without GPGPU cards, provided that they have internet access) a grid of computers or a Cloud.
% paragraph  (end)
% section Introduction (end)

\section{Capabilities and features} % (fold)
\label{sec:Capabilities}

  \subsection{The platform} % (fold)
  \label{sub:subsection name}
  \paragraph{} % (fold)
  \label{par:}
  
    Runs can be distributed over cluster of homogeneous AND heterogeneous machines.
    Distribution can be done locally on the same machine or over the internet (using a embedded island model).
    Parallelization over GPGPU cards leading to massive speedup (x100 to x1000).
    C++-like description language.
  EASEA use CUDA to parallelize over GPGPU card. There is, as absurd as it sound, no
  parallelization over CPU at the time of redaction, but a working openMP prototype
  enters its final testing phase.
  EASEA offer a high level of parametrization and the possibility to include large
  chunk of C/C++ code with little to no change at all.
  % paragraph  (end)
  % subsection subsection name (end)
    \subsection{Current genetic algorithm implementation} % (fold)
    \label{sub:current genetic algorithm implementation}
    \paragraph{} % (fold)
    \label{par:}
    Darwinian and baldwinian approach
    Genetic programming
    CMA-ES (Covariance Matrix Adaptation Evolution Strategy)
    Memetic approach 

    EASEA also offer out-of-the-box a wide range of selector: Max/Min deterministic,
    Max/Min Random, Max/Min tournament and MaxRoulette.
    Three stopping criterion are possible : by generation, by time and by user
    defined control.

    The main advantage of EASEA 
    % paragraph  (end)
    % subsection current genetic algorithm implementation (end)
% section Capabilities (end)


\section{Support} % (fold)
\label{sec:Support}
In order to have a working grapher, java jre 1.6 is required. Without it, an error appears at the start of easea's compiled programs but can be safely ignored.

EASEA had been compiled and tested with those following compilers:

    Gcc 4.4 to 4.8.2
    Clang 3.0 to 3.3
    Mingw-gcc 4.8.2
    CUDA SDK > 4.1

% section Support (end)
\section{Workflow} % (fold)
\label{sec:Workflow}
easea compiler [GP,CUDA,CUDA_GP, etc ...]-> geneticAlgo.ez -> make -> geneticAlgo
(executable) 
% section Workflow (end)

\section{The language} % (fold)
\label{sec:the language}
 \subsection{EASEA defined sections} % (fold)
 \label{sub:EASEA defined sections}
 
Genome specific fields
Genome Initialiser
Description

Uses the Genome EASEA variable to access the individual's genome.
EASEA syntax

\GenomeClass::initialiser :
     ...
\end

Example

\GenomeClass::initialiser :
  for(int i=0; i<SIZE; i++ ) {
    Genome.x[i] = (float)random(X_MIN,X_MAX);
    Genome.sigma[i]=(float)random(0.,0.5);
  }
\end

Genome Crossover
Description

Binary crossover that results in the production of a single child.

The child is initialized by cloning the first parent.

Uses parent1 and parent 2 EASEA variables to access the parents genome.

Uses the child EASEA variable to access the child's genome.
EASEA syntax

\GenomeClass::crossover :
     ...
\end

Example

\GenomeClass::crossover :
  for (int i=0; i<SIZE; i++)
  {
    float alpha = (float)random(0.,1.); // barycentric crossover
    child.x[i] = alpha*parent1.x[i] + (1.-alpha)*parent2.x[i];
  }
\end

Genome mutator
Description

Must return the number of mutations.

Uses the Genome variable to access the individual's genome.
EASEA syntax

\GenomeClass::mutator :
     ...
\end

Example

\GenomeClass::mutator : // Must return the number of mutations
  int NbMut=0; 
  float pond = (float)random(0.,1.);
  for (int i=0; i<SIZE; i++) {
    if (tossCoin(pMutPerGene)){
      NbMut++;
      Genome[i] += pond;
     }
  }
  return NbMut;
\end

Genome Evaluator

The evaluation function is expected to be autonomous and independent from the rest of the code, for correct parallelization.
Description

Must return the fitness of the individual.

Uses the Genome defined variable to access the individual's genome.
EASEA syntax

\GenomeClass::evaluator :
     ...
\end

Example

\GenomeClass::evaluator :
 float Score= 0.0;
 Score= Weierstrass(Genome.x, SIZE);
 return Score;
\end

Genome Display
Description

Uses the Genome variable to access the individual's genome.
EASEA syntax

\GenomeClass::display :
     ...
\end

Example

\GenomeClass::display :
  for( size_t i=0 ; i<SIZE ; i++){
    cout << Genome.x[i] << ":" <<  "|";
  }
  cout << Genome.fitness <<< "\n";
\end



User definition fields
User Declarations
Description

This is the section where the user can declare some global variables and some global function.
EASEA syntax

\User declarations :
     ...
\end

Example

\User declarations :
  #define SIZE 100                              //Problem size
  #define Abs(x) ((x) < 0 ? -(x) : (x))   //Definition of the Abs global function
  #define MAX(x,y) ((x)>(y)?(x):(y))     //Definition of the MAX global function
  #define MIN(x,y) ((x)<(y)?(x):(y))      //Definition of the MIN global function
  #define PI 3.141592654                 //Definition of the PI global variable
\end

User functions
Description

This is the section where the user can declare the different functions he will need.
EASEA syntax

\User declarations :
     ...
\end

Example

\User declarations :
  float gauss()
  /* Generates a normally distributed random value with variance 1 and 0 mean.
      Algorithm based on "gasdev" from Numerical recipes' pg. 203. */
  {
    int iset = 0;
    float gset = 0.0;
    float v1 = 0.0, v2 = 0.0, r = 0.0;
    float factor = 0.0;
    if (iset) {
      iset = 0;
      return gset;
    }
    else {    
      do {
         v1 = (float)random(0.,1.) * 2.0 - 1.0;
         v2 = (float)random(0.,1.) * 2.0 - 1.0;
         r = v1 * v1 + v2 * v2;
      } 
      while (r > 1.0);
      factor = sqrt (-2.0 * log (r) / r);
      gset = v1 * factor;
      iset = 1;
      return (v2 * factor);
    }
  }
\end

User Classes
Description

This is the section where the user will be able to declare:

    The genome of the individuals trough the GenomeClass class
    Different classes that will be needed.

The GenomeClass field has to be present and ideally not empty. All the variables defined in this field (the genome) will be accessible in several other fields using the 'Genome' variable. Other "hidden" variables can be accessed such as:

        The fitness (variable: fitness. Ex: Genome.fitness)

EASEA syntax

\User Class :
     ...
  GenomeClass{
     ...
  }
\end

Example

\User classes:
  Element     { 
    int     Value;
    Element *pNext; }
  GenomeClass { 
    Element *pList; 
    int     Size;   }
\end

User Makefile
Description

This section allows the user to add some flags to the compiler typically to include some custom libraries.

Two flags of the Makefile are accessible to the user in this section:

    CPPFLAGS
    LDFLAGS

EASEA syntax

\User Makefile options :
     ...
\end

Example

\User Makefile options :
  CPPLAGS += -llibrary
  LDFLAGS += -I/path/to/include
\end





Miscellaneous fields
Before everything else function
Description

This function will be called at the start of the algorithm before the parent initialization.
EASEA syntax

\Before everything else function :
     ...
\end

After everything else function
Description

This function will be called once the last generation has finished.

The population can be accessed.
EASEA syntax

\After everything else function :
     ...
\end

At the beginning of each generation function
Description

This function will be called every generation before the reproduction phase.

The population can be accessed.
EASEA syntax

\At the  beginning of each generation function :
     ...
\end

At the end of each generation function
Description

This function will be called at every generation after the printing of generation statistics.

The population can be accessed.
EASEA syntax

\At the end of each generation function :
     ...
\end

At each generation before reduce function
Description

This function will be called at every generation before performing the different population reductions.

The parent population AND the offspring population can be accessed (merged into a single population).
EASEA syntax

\At every generation before reduce function :
     ...
\end




Memetic specific field
Genome Optimiser
Description

The optimiser field is a Genome specific field. It is meant to contain the function defining the way an individual will be locally optimized n times. The function will hence be called sequentially as many times as the user desires for each individual.

EASEA gives the user two possibilities when designing their local optimization function :

    The user can choose to design the function that will enhance the genome of their individuals only in which case the rest of the local optimizer (i.e. creating the local optimizing loop, checking if an individual has improved or not, storing temporary individuals, calling of the evaluation function, etc ...) will be taken care of by the EASEA memetic algorithm. The function will have to be called as many times as specified by the Number of optimisation iterations parameter.
    The user can choose to write the complete local optimizer. This way, he will have the complete freedom to design a more complex and specific optimizer, but he will also have to deal with the creation of the local optimization loop, the management of temporary individuals, the calling of the evaluation function etc... The Number of optimisation iterations parameter will have to be set to 1 as the function desigend by the user will contain it's own optimization loop requiring it's own specific number of optimization iterations.

EASEA syntax

\GenomeClass::optimiser :
     ...
\end

Examples

The two following examples will expose the two different ways the optimizer field can be used. The first example will show a simple mutation function. The second example will show the design of a complete local optimizer. Both examples are GPU compatible.

Genome optimization only

\GenomeClass::optimiser :
 float pas = 0.001;
 Genome.x[currentIteration%SIZE]+=pas;
\end

This example shows a simple mutation function that will add a small variation to one of the genes of an individual. The call to this function will be followed by a call to the evaluation function, and a replacement process. If the modification of the genome has improved the individual, it will replace the original one. This is being taken care of by the EASEA memetic algorithm.

Complete local optimizer

\GenomeClass : : optimiser	:	//	Optimises	the	Genome 
 float pas=0.001;
 float fitnesstmp = Genome.fitness ; 
 float tmp[SIZE]; 
 int index = 0;
 for(int i=0; i<SIZE; i++) 
  tmp[ i ] = Genome.x[ i ];
 for(int i=0; i<100; i++){ 
  tmp[index] += pas;
  fitnesstmp = Weierstrass(tmp, SIZE);
  if(fitnesstmp < Genome.fitness){ 
   Genome. fitness = fitnesstmp ;
   Genome.x[ index ] = tmp[ index ];
  }
  else {
   fitnesstmp = Genome.fitness;
   tmp[ index ] = Genome.x[ index ];
  
   if( pas < 0 )
    index = ( index + 1)%SIZE;
    
   pas *= -1;
  }
 }
\ end

This example shows how to design a complete local optimization function. The genome is almost being changed in the same way as in the first example.
% subsection EASEA defined sections (end)

\subsection{EASEA defined functions} % (fold)
\label{sub:EASEA defined functions}
Random Number Generators
TossCoin
Description

Simulates the toss of a coin. There are two different definitions of the tossCoin function:

    A simple toss of a coin.
    A biased toss of a coin.

EASEA syntax

bool tossCoin()                // SImple tosscoin

bool tossCoin(float bias)  // Biased tossCoin

Example

if( tossCoin(0.1)){
    ...
}

Random
Description

Generates a random number. There are several definitions to the random function:

    Random function with Min Max Boundary
    Random function with Max Boundary only

In the case of a Max boundary only, the Min boundary will be 0.
EASEA syntax

int random( int Min, int Max)

int random( int Max)

float random( float Min, float Max)

double random( double Min, double Max)


% subsection EASEA defined functions (end)
% section the language (end)
\subsection{EASEA defined variable} % (fold)
\label{sub:EASEA defined variable}
Current Generation

Return the current generation number. This variable can be modified.
EASEA syntax

currentGeneration

Number of Generation

Returns the generation limit. This variable can be modified.
EASEA syntax

NB_GEN

Population Size

Returns the size of the population. This variable cannot be modified.
EASEA syntax

POP_SIZE

Mutation Probability

Returns the mutation probability. This variable can be modified.
EASEA syntax

MUT_PROB

Crossover Probability

Returns the crossover probability. This variable can be modified.
EASEA syntax

XOVER_PROB

Minimise

Returns a boolean indicating of the algorithm performs a minimization of not. Returns true if minimizing.
EASEA syntax

MINIMISE

Population

Returns a pointer to the main population. This variable cannot be used everywhere. If misused, it can provoque unexpected behaviours or compile errors.
EASEA syntax

pPopulation ([i] to access individuals)

fitness

pPopulation[i]->fitness

Best individual

Returns a pointer to the best individual found to this point. All the genome field can be accessed as well as the fitness field. All the genome fields can be modified.
EASEA syntax

bBest ("->" operator to access variables)

Genome

Returns the genome of an individual (can only be used in genome specific EASEA sections such as mutation, initialisation, [[EASEA defined sections#Genome Evaluation|evaluation and display). All the fields can be modified.
EASEA syntax

Genome ("." operator to access variables)

Parent 1

Returns the genome of the first selected parent. Can only be used in the crossover genome section.
EASEA syntax

parent1

Parent 2

Returns the genome of the second selected parent. Can only be used in the crossover genome section.
EASEA syntax

parent2

Child

Returns the genome of the newly created individual. Can only be used in the crossover genome section.
EASEA syntax

child




Basic Parameters
Number of generations

Description :
Gives the maximum number of generations during wich the evolutionary algorithm will run.

EASEA syntax :

Number of generations:

Values : Integer strictly over 0.

Command line syntax :

--nbGen=

Values : Integer strictly over 0.
Time limit

Description :
Sets the maximum amount of time in seconds during wich the evolutionary algorithm will be allowed to run. Setting this parameter to 0 deactivates the time limit.

EASEA syntax :

Time limit :

Values : Positive Integer.

Command line syntax :

--timeLimit=

Values : Positive Integer.
Population size

Description :
Sets the size of the population that will be evolved.

EASEA syntax :

Population size :

Values : Integer strictly over 0.

Command line syntax :

--popSize=

Values : Integer strictly over 0.
Offspring size

Description :
Sets the number of individual that will be produced trough evolutionary crossover and/or mutation.

EASEA syntax :

Offspring size :

Values : Integer strictly over 0.

Command line syntax :

--nbOffspring=

Values : Integer strictly over 0.
Mutation probability

Desciption :
This parameter determines the probability that an individual will be mutated during it's creation process.

EASEA syntax :

Mutation probability :

Values : Real number between 0.0 and 1.0 .

Command line syntax :
This parameter is not avaible in command line yet.
Crossover probability

Description :
This parameter determines the probability that an individual will be the result of a crossover during it's creation process.

EASEA syntax :

Crossover probability :

Values : Real number between 0.0 and 1.0 .

Command line syntax :
This parameter is not avaible in command line yet.
Evaluator goal

Description :
This parameter will set the goal of the evolutionary algorithm.

EASEA syntax :

Evaluator goal :

Values : minimis/ze or maximis/ze

Command line syntax :
This parameter is too fundamental to the behaviour of the algorithm to be changed in command line.
Selection operator

Description :
This parameter decides of the way individuals of the parent population will be selected to create the new individuals. Several operator are avaiable in EASEA:

    Tournament (need a selection pressure)
    Deterministic
    Roulette (only when Evaluator goal = maximise)
    Random 

When the selection pressure is an integer over 0, the best individual from the n will be selected.
When the selection pressure is a real number between 0.0 and 1.0, the best individual of 2 will be selection with the probability p given by the selection pressure.

EASEA syntax :

Selection operator: OPERATOR SELECTION_PRESSURE

Values for operators : Tournament Deterministic Roulette Random
Values for selection pressure : Integer strictly over 0 or Real number between 0.0 and 1.0 .

Compile line syntax :

--selectionOperator=
--selectionPressure=

Values for operators : Tournament Deterministic Roulette Random
Values for selection pressure : Integer strictly over 0 or Real number between 0.0 and 1.0 .
Surviving parents

Description :
This parameter will determine the number of children that will be participating in the run for the next generation (IL VA FALLOIR MODIFIER CETTE DESCRITPION)

EASEA syntax :

Surviving offspring:

Values : Integer strictly positive or Percentage (example: 100%)

Command line syntax :

--survivingOffspring=

Values : Integer strictly positive or Real number value between 0.0 and 1.0
Reduce parents operator

Description :
This parameter decides of the way individuals of the parent population will be selected to participate in the next selection process. Several operator are avaiable in EASEA:

    Tournament (need a selection pressure)
    Deterministic
    Roulette (only when Evaluator goal = maximise)
    Random 

When the selection pressure is an integer over 0, the best individual from the n will be selected.
When the selection pressure is a real number between 0.0 and 1.0, the best individual of 2 will be selection with the probability p given by the selection pressure.

EASEA syntax :

Reduce parents operator: OPERATOR SELECTION_PRESSURE

Values for operators : Tournament Deterministic Roulette Random
Values for selection pressure : Integer strictly over 0 or Real number between 0.0 and 1.0 .

Compile line syntax :

--reduceParentsOperator=
--reduceParentsPressure=

Values for operators : Tournament Deterministic Roulette Random
Values for selection pressure : Integer strictly over 0 or Real number between 0.0 and 1.0 .
Reduce offspring operator

Description :
This parameter decides of the way individuals of the offspring population will be selected to participate in the next selection process. Several operator are avaiable in EASEA:

    Tournament (need a selection pressure)
    Deterministic
    Roulette (only when Evaluator goal = maximise)
    Random 

When the selection pressure is an integer over 0, the best individual from the n will be selected.
When the selection pressure is a real number between 0.0 and 1.0, the best individual of 2 will be selection with the probability p given by the selection pressure.

EASEA syntax :

Reduce offspring operator: OPERATOR SELECTION_PRESSURE

Values for operators : Tournament Deterministic Roulette Random
Values for selection pressure : Integer strictly over 0 or Real number between 0.0 and 1.0 .

Compile line syntax :

--reduceOffspringOperator=
--reduceOffspringPressure=

Values for operators : Tournament Deterministic Roulette Random
Values for selection pressure : Integer strictly over 0 or Real number between 0.0 and 1.0 .
Final reduce operator

Description :
This parameter decides of the way individuals will be selected to participate in the next generation. Several operator are avaiable in EASEA:

    Tournament (need a selection pressure)
    Deterministic
    Roulette (only when Evaluator goal = maximise)
    Random 

When the selection pressure is an integer over 0, the best individual from the n will be selected.
When the selection pressure is a real number between 0.0 and 1.0, the best individual of 2 will be selection with the probability p given by the selection pressure.

EASEA syntax :

Final reduce operator: OPERATOR SELECTION_PRESSURE

Values for operators : Tournament Deterministic Roulette Random
Values for selection pressure : Integer strictly over 0 or Real number between 0.0 and 1.0 .

Compile line syntax :

--reduceFinalOperator=
--reduceFinalPressure=

Values for operators : Tournament Deterministic Roulette Random
Values for selection pressure : Integer strictly over 0 or Real number between 0.0 and 1.0 .
Elitism

Description :
This parameter determines the type of elitism of the evolutionary algorithm.

EASEA syntax :

Elitism :

Values : Strong or Weak

Compile line syntax :

---eliteType=

Values : 1 (for Strong) or 0 (for Weak)
Elite

Description :
This parameter sets the number of individuals that are going to be "elites". Set this parameter to 0 to deactivate the elitism

EASEA syntax :

Elite :

Values : Positive Integer. 0 to deactive elitism.

Command line syntax :

--elite=

Values : Positive Integer. 0 to deactive elitism.
Remote Island Model Parameters
Remote island model

Description :
Boolean that will activate the migration of individuals towards other EASEA instances of the same problem that are located on remote machines. The address of the machines has to be specified in the IP file parameter.

EASEA syntax :

Remote island model :

Values : true or false.

Command line syntax :

---remoteIslandModel=

Values : 1 (for true) or 0 (for false).
IP file

Description :
This parameter contains the path to the File containing the IP addresses and port of all the remote machines that can receive individuals. This file needs to be structure as such:
a1.b1.c1.d1:p1
a2.b2.c2.d2:p2
...
an.bn.cn.dn:pn
No blank line is allowed at the end of the file.
An individual can be send to a machine that isn't active.
The IP address and port of the local machine can be present in the file. EASEA will recognize and ignore it.
If there are no IP addresses in the file, or if only the local machine's address is present in that file, the Remote island model Parameter will be set to false.

EASEA syntax :

IP file :

Values : The path to the file containing the IP addresses (example: ./ip.txt)

Command line syntax :

--ipFile=

Values : The path to the file containing the IP addresses (example: ./ip.txt)
Server Port

'Description :
This parameter sets the port that will be used by the island server to listen for the arrival of newcomers.

EASEA syntax :

Server port :

Values : A valid port number.

Command line syntax :

---serverPort=

Values : A valid port number.


Migration probability

Description :
This parameter sets the probability for and individual to migrate to a remote machine every generation.

EASEA syntax :

Migration probability :

Values : A real number between 0.0 and 1.0 .

Command line syntax :

--migrationProbability=

Values : A real number between 0.0 and 1.0 .
Memetic Parameters
Number of optimisation iterations

Description :
This parameters gives the number of iterations the local search algorithm will perform.

EASEA syntax :

Number of optimisation iterations :

Values : A positive integer.

Command line syntax :

--optimiseIterations=

Values : A positive integer.
Baldwinism

Description :
This parameter will set the behaviour of the memetic algorithm. Setting it to true will turn the algorithm into a Baldwinian memetic algorithm. Setting it to false will turn the algorithm into a Lamarckian memetic algorithm. This parameter's default value is true.

EASEA syntax :

Baldwinism :

Values : true or false.

Command line syntax :

--baldwinism=

Values : 1 (for true) or 0 (for false).
Miscalleous Parameters
Print stats

Description :
This parameter set to true will show different stats every generations:

    Generation number
    Time passed since the start of the algorithm
    Number of evaluations completed
    Fitness of the best individual found
    Avegrage fitness
    Standard deviation of the fitness 

This parameter's default value is true.

EASEA syntax :

Print stats :

Value : true or false.

Command line syntax :

printStats=

Values : 1 (for true) or 0 (for false).
Plot stats

Description
This parameter set to true will open a gnuplot process, that will plot the curves representing the evolution of:

    The best fitness
    The avegage fitness
    The standard deviation of the fitness 

To be able to use this parameter, gnuplot has to be installed on the machine. This parameter is only avaiable for Linux users.
The ploting of the curves will happen simultaniously to the evolution. At the end of the evolution, the resulting curves will be saved in a .png file. It's name will be the same as the project name.</br> This parameter's default value is false.

EASEA syntax :

Plot stats :

Value : true or false.

Command line syntax :

plotStats=

Values : 1 (for true) or 0 (for false).
Generate csv stats file

Description :
This parameter set to true will save the different stats shown with the Print stats parameter to a .csv file at the end of the evolution. The file name will be the same as the project name. This parameter's default value is false.

EASEA syntax :

Generate csv stats file :

Values : true or false.

Command line syntax :

--generateCSV=

Values : 1 (for true) or 0 (for false).
Generate gnuplot script

Description :
This parameter set to true will create a gnuplot script that will allow to plot the evolution in a .png file which, name will be the same as the project name. To plot the evolution, the script will save the stats in a .dat file. This parameter's default value is false.

EASEA syntax :

Generate gnuplot script :

Values : true or false.

Command line syntax :

--generateGnuplotScript=

Values : 1 (for true) or 0 (for false).
Generate R script

Description :
This parameter set to true will create a R script that will allow to plot the evolution in a .png file, which name will be the same as the project name. To plot the evolution, the script will save the stats in a .csv file. This parameter's default value is false.

EASEA syntax :

Generate R script :

Values : true or false.

Command line syntax :

--generateRScript=

Values : 1 (for true) or 0 (for false).
Save population

Description :
This parameter set to true will save the final population in a .pop file, which name will be the same as the project name. The population will be in a EASEA serialized form. This parameter's default value is false.

EASEA syntax :

Save population :

Values : true or false.

Command line syntax :

--savePopulation=

Values : 1 (for true) or 0 (for false).
Start from file

Description :
This parameter set to true will initialize the population using the population saved the .pop file that is in the current folder. The file has to have the save name as the project. For this initialization to work, the individuals contained in the file has to have the exact same Genome, and the number of individuals has to be equal or greater to the population size.
The individuals have to be in the EASEA serialization form.
This parameter's default value is false.

EASEA syntax :

Start from file :

Values : true or false.

Command line syntax :

--saveFromFile=

Values : 1 (for true) or 0 (for false).
Print initial population

Description :
This parameter allow to print the initial population. This parameter's default value is 0.

EASEA syntax :
This is a command line exclusive parameter.

Command line syntax :

--printInitialPopulation=

Values : 1 (for true) or 0 (for false).
Print final population

Description :
This parameter allow to print the final population. This parameter's default value is 0.

EASEA syntax :
This is a command line exclusive parameter.

Command line syntax :

--printFinalPopulation=

Values : 1 (for true) or 0 (for false).
User defined parameters

EASEA defines 5 parameters that are left free to be used by the ".ez" programmer.

User parameters :

 --u1 arg                      User defined parameter 1
 --u2 arg                      User defined parameter 2
 --u3 arg                      User defined parameter 3
 --u4 arg                      User defined parameter 4
 --u5 arg                      User defined parameter 5

Types : u1 and u2 are string variables, u3 to u5 are integers.

Usage Exemple : In the .ez source code

  int a = setVariable("u3",b);

where :
"a" is the variable set to the argument or default value.
"u3" is the argument name.
"b" is the default integer value, if the u3 argument has not been set. 

\section{Developer documentation} % (fold)
\label{sec:Developer documentation}
\subsection{Compiler} % (fold)
\label{sub:Compiler}
  lex yacc .... 
% subsection Compiler (end)
  \subsection{EASEA librairy} % (fold)
  \label{sub:EASEA library}
  CEvolutionary ... uml ...
  % subsection EASEA library (end)
% section Developer documentation (end)
% subsection EASEA defined variable (end)

\section{example} % (fold)
\label{sec:example}

% section example (end)

\section{Additional ressources} % (fold)
\label{sec:Additional}
Additional ressource can be found on the project wiki : and repository 
% section Additional (end)
